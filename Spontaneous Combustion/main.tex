\documentclass[12pt]{article}
%\usepackage{report}

\usepackage[utf8]{inputenc} % allow utf-8 input
%\usepackage[T1]{fontenc}    % use 8-bit T1 fonts
\usepackage[colorlinks=true, linkcolor=blue, citecolor=blue, urlcolor=blue]{hyperref}       % hyperlinks
\usepackage{url}            % simple URL typesetting
\usepackage{booktabs}       % professional-quality tables
\usepackage{amsfonts}       % blackboard math symbols
\usepackage{nicefrac}       % compact symbols for 1/2, etc.
%\usepackage{microtype}      % microtypography
\usepackage{lipsum}		% Can be removed after putting your text content
\usepackage{graphicx}
\usepackage{footnote}
\usepackage{doi}
\usepackage{comment}
\usepackage{multirow}
\usepackage{textcomp}
\usepackage{gensymb}
\usepackage{float}
\usepackage{amsmath}
%\usepackage{subfigure}
\usepackage{subcaption}
\usepackage{setspace}
\usepackage[skip=10pt plus1pt]{parskip} 
\usepackage[top=3cm, bottom=3cm, left=2.5cm, right=2.5cm]{geometry}
\usepackage{titlesec}
\begin{document}
\onehalfspacing
\begin{titlepage}
    \centering
    \includegraphics[width=3cm]{crest.jpg}\par
    \vspace{0.3cm}
    {\scshape\Large School of Mathematical Sciences \par}
    \vspace{0.25cm}
    {\scshape\Large The University of Southampton \par}
    \vspace{0.25cm}
    {\Large MATH 6149 - Modelling with Differential Equations \par}
    \vspace{0.5cm}
    {\huge\bfseries Spontaneous Combustion \par}
    \vspace{0.5cm}

    {\Large Chin Phin Ong (Linus) \par}
    \vspace{0.25cm}
    {\large Student ID: 33184747 \par}
    {\large  \par}
    \vfill
    {\large March 2025 \par}
\end{titlepage}

\section{Introduction}
Spontaneous combustion is not something the average Joe encounters on a day to day basis. This phenomenon occurs when flammable material goes up in flames without an external source of heat, rather it occurs via internal chemical reactions \cite{SPEIGHT2011355}. These chemical reactions happen slowly, producing heat greater than the material can dissipate it, until the temperature rises to a point of combustion (thermal runaway). 

In this report, I will introduce in detail what spontaneous combustion is, and how one can model this frightening phenomenon using differential equations. Finally, I will briefly touch on what the solutions given by solving the differential equations mean, and how to avoid spontaneous combustion.

\section{Spontaneous Combustion}
As mentioned, spontaneous combustion just does not happen in your backyard, unless your backyard is a 50 acre plot of land with mountains of hay, or you live with a scurry of squirrels who have accumulated way too many pistachios (also who happen to have "oil-soaked fibres" nearby \cite{TransportInformationService}).

To start, we can derive a conservation law. Consider a material with infinitesimal thickness $dx$, infinitesimal volume $dV$ and a single side's surface area of $A$. Heat flows from $x$ to $x + dx$. The heat energy $H$ in a material per unit volume is:
\begin{equation}
    H = \rho c_\rho T
\end{equation}
Where $\rho$ and $c_\rho$ are the density and the specific heat capacity of the material respectively, with $T$ being the temperature. The heat energy at time $t$ is simply $H\cdot dV$, and the flux into $dV$ is:
\begin{equation}
    \text{Net heat flow into } dV = \underbrace{q(x, t)A}_{\text{flow in}} - \underbrace{q(x + dx, t)A}_{\text{flow out}}.
\end{equation}

Since we are considering the internal reactions that generate heat, the heat generation in $dV = G \cdot A \cdot dx$, where $G$ is the rate of heat generation. If we do a first order Taylor expansion for $q(x + dx, t)$, 
\begin{equation}
    q(x + dx, t) \approx q(x, t) + \frac{\partial q}{\partial x}\cdot dx
\end{equation}
The net heat flow is then: $- \frac{\partial q}{\partial x}\cdot A$. 

This expression can be used to get closer to our final expression.
\subsection{Conservation of Energy}
The rate of change of heat energy in $dV$ must equal the net flux plus any internal heat generation, i.e.:
\begin{equation}
\begin{split}
    \rho c_\rho \cdot A \cdot  dx\frac{\partial T}{\partial t} + \frac{\partial q}{\partial x}\cdot A \cdot  dx &= G\cdot A \cdot  dx
    \\\rho c_\rho\frac{\partial T}{\partial t} + \frac{\partial q}{\partial x}&= G
\end{split}    
\label{eqn:conservation-law}
\end{equation}
This is the thermal energy conservation law. Now to relate heat flux $q$ and temperature, we can introduce Fourier's law:
\begin{equation}
    q = -k \frac{\partial T}{\partial x}
\end{equation}
With $k$ being the thermal conductivity of the material. We can substitute Fourier's law into equation \ref{eqn:conservation-law}, obtaining the 1D diffusion equation with source terms on the right:
\begin{equation}
    \boxed{\space \rho c_\rho\frac{\partial T}{\partial t} = \frac{\partial}{\partial x} \left(k \frac{\partial T}{\partial x}\right) + G \space}
\end{equation}
\section{Modelling}

\bibliography{references}
\bibliographystyle{plain}
\end{document}